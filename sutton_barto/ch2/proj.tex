%\documentclass[12pt]{report}

\documentclass[12pt]{report}

\usepackage[margin=1.0in]{geometry} % margins
\usepackage{amsmath}
\usepackage{amsfonts}
\usepackage{amsthm} % theorem formatting; proof environments
\usepackage{MnSymbol} % \blacksquare
\usepackage{tikz-cd} % commutative diagrams
\usepackage[framemethod=TikZ]{mdframed}


\DeclareMathOperator*{\argmax}{arg\,max}
\DeclareMathOperator*{\argmin}{arg\,min}

\newcommand*{\vertbar}{\rule[-1ex]{0.5pt}{2.5ex}}
\newcommand*{\horzbar}{\rule[.5ex]{2.5ex}{0.5pt}}



  
  
%\newtheorem{env_name}{Word_to_Print}[hier]
\newtheorem{mythm}{Theorem}[chapter] % restart counter at each chapter
\newtheorem{prop}[mythm]{Prop}% use same counter as theorem env
\newtheorem{corollary}{Corollary}[mythm] % restart counter when theorem is used
\newtheorem{lemma}[mythm]{Lemma}
% \begin{theorem}[Pythagorean Theorem] uses text
\renewcommand\qedsymbol{$\blacksquare$}

\newenvironment{problem}[2][Problem]{\begin{trivlist}
\item[\hskip \labelsep {\bfseries #1}\hskip \labelsep {\bfseries #2.}]}{\end{trivlist}}
\newenvironment{exercise}[2][Exercise]{\begin{trivlist}
\item[\hskip \labelsep {\bfseries #1}\hskip \labelsep {\bfseries #2.}]}{\end{trivlist}}

\newenvironment{theorem}%
  {\begin{mdframed}[backgroundcolor=black!10,
  	linewidth=0pt,]\begin{mythm}}%
  {\end{mythm}\end{mdframed}}

\surroundwithmdframed[linewidth=3pt,
   topline=false,
   rightline=false,
   bottomline=false,
   leftmargin=\parindent,
   skipabove=\medskipamount,
   skipbelow=\medskipamount,
]{proof}





\newcommand*{\bit}{\begin{itemize}}
\newcommand*{\eit}{\end{itemize}}


\title{Sutton \& Barto Reinforcement Learning Notes}
\author{Stan Tuznik}
  
\begin{document}

\maketitle
\tableofcontents


\chapter{Introduction}

\chapter{Multi-armed Bandits}

\bit
	\item Reinforcement learning evaluates states and actions, it doesn't just tell what action to take.
	\bit
		\item Evolutionary methods, by contrast, take a policy, evaluate it as a whole, and modify it. It does this holistically, not by inspecting and evaluating individual actions!
		\item Evaluating actions requires \emph{taking} them; this requires exploration.
	\eit
	\item \textbf{Bandit problems} use only a single state in which one of $k$ actions may be taken. The goal is to evaluate each action.
\eit

\section{A $k$-armed Bandit Problem}
\bit
	\item \textbf{Bandit problem}: we have $k$ possible actions, and we can take one action in each of $N$ time steps. Taking actions results in an immediate reward. \textbf{Goal} is to learn how to behave in order to maximize the total reward over all time steps.
	\item When an action $A_t$ is taken, a reward $R_t$ is obtained. In general, this reward will be a random variable, and so comes from some unknown distribution, which we assume is \textit{stationary} over time.
	\bit	
		\item We can consider the expected reward of taking action $A_t = a$: 
		\begin{equation}
		q_*\left(a\right) := \mathbb{E}\left[R_t \, \middle| \, A_t = a\right]
		\end{equation}
		We call this the \textbf{value of taking action $a$ at time $t$}.
		\item The function $q_*: \mathcal{A} \rightarrow \mathbb{R}$ so defined is the \textbf{action-value function}.
	\eit
	\item If we knew this action-value function, we could ``beat probability'' by always choosing the action with the highest expected reward, i.e., the highest action-value.
	\item Unfortunately, we don't actually know what the distribution of rewards looks like, so we can't exactly compute this action-value function.
	\bit
		\item Instead, we approximate it with a function $Q_t$ such that $Q_t\left(a\right)$ is our estimated value of action $a$ at time step $t$.
		\bit
			\item Does this mean the estimated value of taking action $a$ at time $t$, or the estimated value at time $t$ of taking action $a$? It seems from the notation ``$q_*$'' not indicating $t$ that it must be the latter.
			\item We want $Q_t\left(a\right) \approx q_*\left(a\right)$, so this seems to support my conclusion.
			\item We also assumed that the distributions are stationary and so the value of any particular action should not depend on when we take it.
		\eit
	\eit
	\item A \textbf{greedy action} is one with the maximum expected total reward; an action is \textbf{exploratory} if it is not greedy.
	\item Strategizing about whether to act greedily depends on uncertainties, estimates, and the finite time horizon (if we don't have much time left, it's probably not worth exploring).
	
\eit 


\section{Action-value Methods}

\bit
	\item \textbf{Action-value methods} estimate the value of actions and use those estimates to make action selection decisions.
	\bit
		\item The \emph{true value} of an action is the expected (mean) reward when that action is taken. We can't possibly know this, though!
	\eit
	\item One simple approach is the \textbf{sample-average} method, where take a large number ($t-1$) of actions, observe the reward obtained each time, and compute the empirical average reward observed after taking action $a$: 
	\begin{equation}
		Q_t\left(a\right) := \frac{\sum_{i=1}^{t-1} R_i \cdot \mathbb{I}_{A_i=a}}{\sum_{i=1}^{t-1}\mathbb{I}_{A_i=a}}	
	\end{equation}
	\bit
		\item This actually converges to $q_*$ in the limit as $t \rightarrow \infty$.
	\eit
	\item How can we use these estimates to select actions?
	\bit
		\item The greedy action is the action (or set thereof) which maximizes the expected reward: 
		\begin{equation}
			A_t = \argmax_a Q_t\left(a\right)		
		\end{equation} 
		\item ``Greediness always exploits current knowledge to maximize immediate reward.''
		\item Performing greedily is entirely exploitation and no exploitation!
	\eit
	\item \textbf{$\epsilon$-greedy methods} incorporate exploration by setting a parameter $\epsilon \in \left(0,1\right)$ --- generally to a small value --- and taking an exploratory action $\epsilon \times 100 \%$ of the time.
	\bit
		\item This method theoretically guarantees that all actions will be sampled an infinite number of times, so that $Q_t\left(a\right) \rightarrow q_*\left(a\right)$.
	\eit
	
\eit



\section{The 10-armed Testbed}

\bit
	\item The authors build a simple bandit test problem. There are $k=10$ actions to take, and for each action, the rewards are normally distributed. 
	\bit
		\item Thus, taking an action multiple times can give different rewards, but the rewards all come from the same distribution.
		\item The authors use a simple normal distribution for the reward distribution for each action.
		\begin{equation}
			r \sim \text{Pr}\left(r\,;\, A=a\right) = \mathcal{N}\left(q_*\left(a\right), 1\right)
		\end{equation}
		where the action-values $q_*\left(a\right)$ are generated from a standard normal distribution:
		\begin{equation}
			q_*\left(a\right) \sim \mathcal{N}\left(0,1\right)
		\end{equation}
		\item This is fairly easy to implement, and gives us a ``black box'' environment for experimenting with different bandit algorithms, such as the aforementioned $\epsilon$-greedy method with different values of $\epsilon$.
	\eit
\eit


\end{document}