\documentclass[12pt]{report}

\usepackage[margin=1.0in]{geometry} % margins
\usepackage{amsmath}
\usepackage{amsfonts}
\usepackage{amsthm} % theorem formatting; proof environments
\usepackage{MnSymbol} % \blacksquare
\usepackage{tikz-cd} % commutative diagrams
\usepackage[framemethod=TikZ]{mdframed}


\DeclareMathOperator*{\argmax}{arg\,max}
\DeclareMathOperator*{\argmin}{arg\,min}

\newcommand*{\vertbar}{\rule[-1ex]{0.5pt}{2.5ex}}
\newcommand*{\horzbar}{\rule[.5ex]{2.5ex}{0.5pt}}



  
  
%\newtheorem{env_name}{Word_to_Print}[hier]
\newtheorem{mythm}{Theorem}[chapter] % restart counter at each chapter
\newtheorem{prop}[mythm]{Prop}% use same counter as theorem env
\newtheorem{corollary}{Corollary}[mythm] % restart counter when theorem is used
\newtheorem{lemma}[mythm]{Lemma}
% \begin{theorem}[Pythagorean Theorem] uses text
\renewcommand\qedsymbol{$\blacksquare$}

\newenvironment{problem}[2][Problem]{\begin{trivlist}
\item[\hskip \labelsep {\bfseries #1}\hskip \labelsep {\bfseries #2.}]}{\end{trivlist}}
\newenvironment{exercise}[2][Exercise]{\begin{trivlist}
\item[\hskip \labelsep {\bfseries #1}\hskip \labelsep {\bfseries #2.}]}{\end{trivlist}}

\newenvironment{theorem}%
  {\begin{mdframed}[backgroundcolor=black!10,
  	linewidth=0pt,]\begin{mythm}}%
  {\end{mythm}\end{mdframed}}

\surroundwithmdframed[linewidth=3pt,
   topline=false,
   rightline=false,
   bottomline=false,
   leftmargin=\parindent,
   skipabove=\medskipamount,
   skipbelow=\medskipamount,
]{proof}


